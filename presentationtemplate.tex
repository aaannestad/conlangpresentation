\documentclass{article}

\usepackage{fontspec}
\setmainfont{Coelacanth}[Numbers=OldStyle]

\usepackage{geometry}[margins=2cm]

\begin{document}

This is a document meant to serve as a template for presenting a conlang as a piece of art. It's meant to help you generate a document that shows the structure of a conlang clearly and engagingly without overwhelming a reader with too much information. It is not meant as a template for writing a grammar, though it is organised along similar lines; the goal here is a quick and concise presentation rather than a full description. Any relevant question should be answered with one or two sentences and an example or two, with maybe a few questions answered more in-depth when they touch on unique and interesting aspects of the language.

This document is meant as a guide for writing a presentation, and very much \emph{not} as a guide for creating a conlang in the first place. You should not create your conlang with the explicit purpose of generating interesting responses to these questions! Nonetheless, you may find it a useful guide to the kinds of questions you'll need to be able to answer when creating a language. You should feel free to depart from the guidelines given here whenever doing so would better showcase your language; given that it's impossible to write a grammar template that can handle any language, this presentation template will also fail to handle every language---it's meant to focus on questions most naturalistic languages will need to answer, but it certainly doesn't cover every situation even a naturalistic language might end up in. Treat it as a general guide, not a questionnaire you need to turn in. But don't forget the purpose: this is meant to generate a \emph{quick and punchy} presentation of your language, not a full description of its every working.

\hfill Aidan Aannestad

\hfill v0.1; \today

\newpage

\section{Introductory questions}

\begin{itemize}
  \item What's the purpose of this conlang? Is it meant for a fictional setting, or an exploration of linguistic mechanics, or a personal language for personal use, or something else?
  \item If this language has a fictional setting, what are some important details about the sociolinguistic situation it's set in? Is it in contact with any other language?
  \item Does this language have an in-world diachronic history?
\end{itemize}

\section{Phonology}

\subsection{Phoneme inventory}
\begin{itemize}
  \item Give a table or two of your language's phonemic inventory, organised in such a way as to highlight the phonological contrasts your language makes use of. You can mostly ignore phonetic detail here---for example, if your language has no palatals besides /j/ and no alveolar approximant, you can probably just put /j/ in the same column as /t d/ and so on.
  \item Are any of the above phonemes marginal, or questionable in some way? What arguments can be made for or against their status as phonemes?
  \item Are there any interesting distributional restrictions on certain sounds?
\end{itemize}

\subsection{Syllable structure and word shape}
\begin{itemize}
  \item What is the maximal syllable structure your language allows? How many moras can a syllable have?
  \item Are there phonemically long vowels? Long consonants? Can you have both in one syllable? Are diphthongs the same as long vowels in terms of weight?
  \item What's the minimum size a root can have? What's the most basic and iconic shape roots have? Do certain word classes have certain shapes associated with them?
  \item Does your morphology naturally result in words that follow the above syllable structure, or do you need to process inflected words before they have legal surface forms? What repair processes do you have?
\end{itemize}

\subsection{Suprasegmentals}
\begin{itemize}
  \item Does your language have a stress system? If so---
        \begin{itemize}
          \item How is stress normally assigned? What's the basic prosodic foot type? Can you have extrametrical moras or syllables?
          \item What are the normal phonetic correlates of stress? How does a listener know a syllable is stressed?
          \item Do unstressed syllables have fewer phonological contrasts available than stressed ones?
          \item Does stress move when inflectional morphology is added?
        \end{itemize}
  \item Does your language have a tone system? If so---
        \begin{itemize}
          \item How many different tone levels do you have? What are the phonemic melodies those levels can combine into?
          \item How is tone assigned? What's the tone-bearing unit, the mora or the syllable? What happens when you have too many tones for the number of tone-bearing units in a word, or too few?
          \item Do tones ever get downstepped or upstepped?
          \item Do you ever end up with floating tones? Do you have any morphemes that have no segmental content and are only a floating tone?
          \item Do you have automatic boundary tones on one or the other side of prosodic phrases?
        \end{itemize}
  \item If you have both a stress system and a tone system, how do they interact?
  \item Do you have any other properties besides tone that behave autosegmentally? Nasalisation, vowel features, phonation...?
\end{itemize}

\section{Word classes}%Help me please, Logan :P
\begin{itemize}
  \item How many major open word classes do you have?
  \item What word classes can stand as predicates?
  \item What, if any, word classes can modify other words directly? Do you have separate classes for words that modify referential words like nouns and words that modify predicative words like verbs?
\end{itemize}

\section{Morphology}
\begin{itemize}
  \item How many categories does a predicative word inflect for? How many of those have independent affixes, and how many are fused together? Do these affixes follow a strict ordering template, or can you reorder them to get different scoping effects?
  \item Can you have more than one lexical root inside a predicative word?
  \item How many categories does a referential word inflect for?
  \item How do inflectional morphemes interact with each other? Do you simply line them up in order and let the phonology take over, or are there morphophonemic alternations?
  \item What's the citation form of each inflectable word class? Can you know all the inflected forms of a word from its citation form, or do you need to memorise several? If so, which ones, and why are they unpredictable?
  \item Do you have any nonlinear morphology, like ablaut or infixes? Do you have any template-based morphology?
  \item What does your personal pronoun system look like? How many grammatical categories do your pronouns carry? Do you have real third-person pronouns, or do you just use demonstratives?
  \item How many degrees of deixis do your demonstrative pronouns show? How many different categories of demonstrative/interrogative/indefinite/etc pronouns do you have? Do you derive one kind of pronoun from another?
\end{itemize}

\section{Sentence-internal syntax}
\begin{itemize}
  \item What word order does a basic predicate-focus sentence follow? Where do modifiers come in such a sentence?
  \item Where do phrases that modify referential words come relative to their heads? If you can place them on either side, what motivates the choice of one side or the other?
  \item What morphosyntactic alignment does your core case marking follow? What morphosyntactic alignment does your underlying syntax follow for the purposes of things like cross-clause coreference? Is it the same for every predicate, or do some kinds of predicates have a different alignment pattern?
  \item Can you drop highly discourse-active referents entirely, or do you need to have at least a pronoun filling every argument slot a given predicate requires?
  \item How do you convert a clause to a modifier (i.e\ how do you make a relative clause)? What arguments can be gapped? How are gaps handled inside the clause?
  \item How are clauses joined? Do you use primarily separate conjunction words (and if so, are some of them the same as reference phrase conjunctions?), or do you use mostly verb morphology?
  \item Do you handle adverbial clauses as subordinate clauses, or via the same mechanism as coordinated clauses? Or do you use clause chaining instead?
  \item How does your language track referents between clauses? Do you have to keep them in the same syntactic position in each clause, like in English? Does your conjunction morphology keep track of which is in that syntactic position? Do you just use noun class agreement? Do you rely entirely on contextual inference?
  \item How do you handle control and raising situations?%this probably needs MASSIVE expansion
\end{itemize}

\section{Information structure}
\begin{itemize}
  \item How does your language mark topics? Do you have a specific strategy for marking topics, or does topicality have to be inferred from other properties like definiteness or subjecthood?
  \item How does your language mark contrastive topics?
  \item How does your language mark the five different focus domain types (predicate focus, argument focus, sentence focus, verb focus, verum focus)? Do you have specific morphological marking involved in any of them?
  \item How does your language distinguish between different kinds of focus?
  \item How does your language mark frame setters? Do you just use word order, or do you reuse topic marking or focus marking, or do you have a separate way of handling them?
  \item How does your language introduce wholly new referents? How does it reactivate previous topics that haven't come up in a while?
\end{itemize}

\section{Discourse}%super WIP
\begin{itemize}
  \item What are the different types of discourses your language has? Human languages need ways to 1) tell stories about the past, 2) describe and explain the present, 3) lay out how to perform an action, and 4) exhort people to behave in a certain way; how do the categories your language uses fit into these? Do any subdivide into further categories? Are any treated the same way?
  \item What tense and/or aspect do you use for the main narrative line in each type of discourse? What tenses and/or aspects do you use in background information? Is there any other grammar that helps distinguish the main line from other parts of the discourse?
  \item What clause connection strategies do you use to chain together events in the main line of each narrative? What clause connection strategies do you use to attach background information?
  \item Does the climax of a type of discourse behave differently in the above ways than the rest of the discourse?
\end{itemize}

\end{document}